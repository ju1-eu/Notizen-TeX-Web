% ju 11-Aug-20
Erstellt Websiten \& Latex-Files mit Markdown und Pandoc. Projekt wurde
getestet unter >>Ubuntu 18.04.3 LTS<< und >>Win10<< (erfordert
\textbf{Git Bash})

\section{Software}\label{software}

\begin{itemize}
\item
  Git Bash\footnote{\url{https://git-scm.com/downloads}}
\item
  Github-Repository klonen\footnote{\url{https://github.com/ju1-eu/Vektorgrafiken-SVG-EPS.git}}
\item
  Texlive (Latex)\footnote{\url{https://www.tug.org/texlive/}}
\item
  Pandoc (Dokumentenconverter)\footnote{\url{https://pandoc.org/installing.html}}
\item
  Imagemagick (Bildbearbeitung)\footnote{\url{https://imagemagick.org/script/download.php}}
\item
  Editor Visual Studio Code\footnote{\url{https://code.visualstudio.com/}}
\item
  Editor Atom\footnote{\url{https://atom.io/}}
\item
  Editor Notepad++\footnote{\url{https://notepad-plus-plus.org/downloads/}}
\item
  TeXstudio (Latexeditor)\footnote{\url{https://www.texstudio.org/}}
\item
  Tablesgenerator (Latex / Markdown)\footnote{\url{https://www.tablesgenerator.com/latex_tables}}
\item
  hpi-dokumentvorlagen-latex (Hasso-Plattner-Institut (HPI)
  Potsdam)\footnote{\url{https://osm.hpi.de/theses/tipps\#dokumentvorlagen-latex}}
\item
  Zotero (Literaturverwaltung)\footnote{\url{http://www.zotero.org/}}
\item
  Wordpress\footnote{\url{https://de.wordpress.org/download/}}
\item
  XAMPP Apache + MariaDB + PHP\footnote{\url{https://www.apachefriends.org/de/index.html}}
\item
  Filezilla\footnote{\url{https://filezilla-project.org/}}
\item
  VM VirtualBox\footnote{\url{https://www.virtualbox.org/}}
\item
  Ubuntu (Desktop / Server)\footnote{\url{https://ubuntu.com/download}}
\item
  Wordpress-themes\footnote{\url{https://de.wordpress.org/themes/}}
\item
  themecheck (Wordpress-themes)\footnote{\url{https://themecheck.info/}}
\item
  ghostscript Z.B eps in pdf\footnote{\url{https://www.ghostscript.com/}}
\end{itemize}

\section{Erste Schritte}\label{erste-schritte}

\textbf{Files anpassen:}

\begin{enumerate}
\item
  \verb|scripteBash/sed.sh|

  \begin{itemize}
  \item
    codelanguage:
    \verb|HTML5, Python, Bash, C, C++, TeX|
  \item
    CMS Server Pfad: \verb|https://bw-ju.de/#|
  \item
    Bildformat: svg, png, jpg, webp
  \end{itemize}
\item
  \verb|scripteBash/gitversionieren.sh|

  \begin{itemize}
  \item
    >>/media/jan/usb/repos/notizenUbuntu<<
  \item
    >>/media/jan/virtuell/repos/notizenUbuntu<<
  \end{itemize}
\item
  \verb|projekt.sh|

  \begin{itemize}
  \item
    THEMA=>>Vektorgrafiken-SVG-EPS<<
  \item
    >>/media/jan/usb/backup/notizenUbuntu<<
  \item
    >>/media/jan/virtuell/backup/notizenUbuntu<<
  \item
    >>/media/jan/usb/archiv/notizenUbuntu<<
  \item
    >>/media/jan/virtuell/archiv/notizenUbuntu<<
  \end{itemize}
\item
  \verb|content/metadata.tex|

  \begin{itemize}
  \item
    Datum, Titel, Autor
  \end{itemize}
\item
  \verb|content/titelpage.tex|

  \begin{itemize}
  \item
    >>Grafiken/logo.pdf<<
  \end{itemize}
\end{enumerate}

\textbf{Markdown-Files erstellen}

\begin{enumerate}
\item
  Erstelle eine Datei >>neu.md<< im Ordner >>md/<<

  \begin{itemize}
  \item
    Bilder nach \verb|images/| kopieren
  \item
    Vektorgrafiken nach \verb|Grafiken/| kopieren
  \end{itemize}
\item
  Script ausführen: \verb|projekt.sh|
\end{enumerate}

\textbf{Linux-Terminal} öffnen oder unter Win10 \textbf{Git
Bash-Terminal} öffnen

\lstset{language=C}% C, TeX, Bash, Python 
\begin{lstlisting}[
	%caption={}, label={code:}%% anpassen
]
$ ./projekt.sh

  0) Projekt aufräumen
  1) Projekt erstellen
  2) Markdown in (tex, html5) + sed (Suchen/Ersetzen)
  3) Kapitel erstellen + Scripte ausführen
  4) Fotos optimieren (Web, Latex)
  5) www + index.html
  6) git init
  7) git status + git log
  8) Git-Version erstellen
  9) Backup + Archiv erstellen
 10) Beenden?

 Eingabe Zahl >_
\end{lstlisting}

\begin{enumerate}
\setcounter{enumi}{2}
\item
  Latex-PDFs erstellen: \verb|make|
\end{enumerate}

\lstset{language=C}% C, TeX, Bash, Python 
\begin{lstlisting}[
	%caption={}, label={code:}%% anpassen
]
$ make
$ make clean
$ make distclean
\end{lstlisting}

\begin{enumerate}
\setcounter{enumi}{3}
\item
  Repository auf Github erstellen
\end{enumerate}

\section{Github-Repository erstellen --
klonen}\label{github-repository-erstellen-klonen}

GitHub's maximum file size of \textbf{50 MB}

\textbf{Repository auf Github erstellen}

\lstset{language=C}% C, TeX, Bash, Python 
\begin{lstlisting}[
	%caption={}, label={code:}%% anpassen
]
# HTTPS oder SSH
HTTPS: https://github.com/ju1-eu/Vektorgrafiken-SVG-EPS.git
SSH: git@github.com:ju1-eu/Vektorgrafiken-SVG-EPS.git

# create a new repository 
echo "# README" >> README.md
git init
git add .
git commit -m "git init"
                
# or push an existing repository 
git remote add origin https://github.com/ju1-eu/Vektorgrafiken-SVG-EPS.git
git push -u origin master
\end{lstlisting}

\textbf{Github-Repository klonen}

\lstset{language=C}% C, TeX, Bash, Python 
\begin{lstlisting}[
	%caption={}, label={code:}%% anpassen
]
git clone https://github.com/ju1-eu/Vektorgrafiken-SVG-EPS.git
\end{lstlisting}

\section{Script Beschreibung}\label{script-beschreibung}

\verb|$ ./projekt.sh|

\begin{enumerate}
\item
  Projekt erstellen

  \begin{itemize}
  \item
    Verz. erstellen, wenn nicht vorhanden
  \end{itemize}
\item
  Markdown in \verb|*.tex und *.html|

  \begin{itemize}
  \item
    Markdown in Latex + HTML5 + Wordpress
  \item
    sed > Wordpress
  \item
    sed > Latex
  \end{itemize}
\item
  Kapitel erstellen + Scripte ausführen

  \begin{itemize}
  \item
    Alle Abbildungen >>images/<< in Markdown speichern.

    \begin{itemize}
    \item
      >>archiv/input-img.txt<<
    \end{itemize}
  \item
    Latex Kapitel erstellen.

    \begin{itemize}
    \item
      Kopiere >>tex-pandoc/.tex<< nach >>tex/<<
    \item
      >>tex/<< \textbf{Handarbeit\ldots{}} für opt. Ergebnisse!
    \item
      Kopiere >>archiv/inhalt.tex<< nach >>content/<<
    \item
      make -- Latex-PDF erstellen
    \end{itemize}
  \item
    Tabellen als PDFs in Latex einfügen. >>Tabellen/ ?<<
  \item
    Inhalt vom Projektverzeichnis.

    \begin{itemize}
    \item
      >>archiv/Projekt-Inhalt.txt<<
    \end{itemize}
  \item
    Quellcode >>code/<< in Latex speichern.

    \begin{itemize}
    \item
      >>archiv/Quellcode-files.tex<< HTML, Python, Bash, C, C++, TeX
    \end{itemize}
  \item
    Artikel aus den Ordnern erstellen

    \begin{itemize}
    \item
      >>tex/<<
    \item
      >>archiv/<<
    \item
      >>Tabellen/<<
    \item
      >>content/beispiele/tex/<<
    \item
      wird gespeichert in >>Artikel/<<
    \end{itemize}
  \item
    Alle Abbildungen >>images/<< in Latex speichern

    \begin{itemize}
    \item
      >>archiv/Pics-files.tex<<
    \item
      Bildgröße: \verb|width=.80\\textwidth|
    \end{itemize}
  \end{itemize}
\end{enumerate}

\begin{enumerate}
\setcounter{enumi}{3}
\item
  Fotos optimieren (Web, Latex)
\item
  www + index.html

  \begin{itemize}
  \item
    >>html/alle-pics.html<< erstellen
  \item
    >>index.html<< erstellen
  \end{itemize}
\item
  \verb|git init|
\item
  \verb|git status| +
  \verb|git log|
\item
  Git-Version erstellen

  \begin{itemize}
  \item
    \textbf{Pfade} anpassen in
    \verb|gitversionieren.sh|
  \item
    lokales Repository: master
  \item
    Github Repository: origin/master
  \item
    Backup Repository: backupUSB/master

    \begin{itemize}
    \item
      >>/media/jan/usb/repos/notizenUbuntu<<
    \end{itemize}
  \item
    Backup Repository: backupHD/master

    \begin{itemize}
    \item
      >>/media/jan/virtuell/repos/notizenUbuntu<<
    \end{itemize}
  \end{itemize}
\item
  Backup + Archiv erstellen

  \begin{itemize}
  \item
    \textbf{Pfade} anpassen in \verb|projekt.sh|
  \item
    THEMA=>>Vektorgrafiken-SVG-EPS<<
  \item
    >>/media/jan/usb/backup/notizenUbuntu<<
  \item
    >>/media/jan/virtuell/backup/notizenUbuntu<<
  \item
    >>/media/jan/usb/archiv/notizenUbuntu<<
  \item
    >>/media/jan/virtuell/archiv/notizenUbuntu<<
  \end{itemize}
\end{enumerate}
